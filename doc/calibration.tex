\documentclass {article}

\usepackage{graphicx}
\usepackage{amssymb}
\usepackage{amsmath}
\newcommand\conf{\mathbf{q}}
\newcommand\confoffset{\mathbf{q}_{off}}
\newcommand\transf[2]{^{#1}M_{#2}}
\newcommand\se[1]{\mathfrak{se}(#1)}
\newcommand\transl{\mathbf{t}}
\newcommand\linvel{\mathbf{v}}
\newcommand\angvel{\omega}
\newcommand\cross[1]{\left[#1\right]_{\times}}
\newcommand\x{\mathbf{x}}
\newcommand\y{\mathbf{y}}
\newcommand\hole{\mathbf{h}}
\newcommand\tool{\mathbf{t}}
\newcommand{\reals}{\mathbb{R}}

\title {Calibration of Pyrene arms and camera}
\author {Florent Lamiraux}
\date {}
\begin{document}
\maketitle
\section{Introduction}

\begin{figure}
  \centerline{
    \includegraphics[width=.7\linewidth]{end-effector.png}
  }
  \caption{End effector with the tooltip and camera frames.}
  \label{fig:end-effector}
\end{figure}

The objective of this note is to process data collected with the UR10e robot
equipped with a camera in order to calibrate the pose of the camera and the position of the tooltip in the end effector frame.

\section{Calibration procedure}

The calibration procedure consists in two successive steps:
\begin{enumerate}
\item calibrating the pose of the camera in the end effector frame,
\item calibrating the pose of the tooltip in the end effector frame.
\end{enumerate}

\subsection{Notation}

We denote by
\begin{itemize}
\item $\conf_1,\cdots,\conf_N$, $N$ calibration configurations,
\item $\transf{e}{c}\in SE(3)$ the pose of the camera in the end-effector frame. This
  is the unknown of the first step,
\item $\hat{\transf{c}{o}}^i\in SE(3)$ the pose of the part in the camera frame measured by the part localization software when the robot is in configuration $\conf_i$,
\item $^{o}\hole\in\reals^3$, the position of the hole in the part frame,
\item $\hat{^{c}\hole}^i = \hat{\transf{c}{o}}^i\, ^{o}\hole\in\reals^3$ the position of the hole in the camera frame, as measured by the localisation software,
\item $^{e}\tool$, the nominal position of the tooltip in the end-effector frame (without offset applied),
\item $^{e}\hat{\Delta}_i$ the offset applied to align the tooltip with the hole in configuration $\conf_i$, expressed in the end-effector frame.
\end{itemize}

\subsection{Pose of the camera in the end effector frame}

The first step consists in generating a motion where the tool rotates in front
of a hole, stopping at 4 different orientations.

For each orientation, we use the \textit{tooltip calibration widget} to align
the tooltip with the hole and we save the data in a file.

For each measurement, we get the following equation:
$$
\transf{c}{e}(^e\tool + ^{e}\hat{\Delta}_i) - \hat{^{c}\hole}^i = 0
$$
The Jacobian of this equation with respect to $\transf{c}{e}$ is:
%% \begin{eqnarray*}
%%   \frac{d}{dt}(\transf{c}{e}(^e\tool + ^{e}\hat{\Delta}_i)) &=&
%%   \left(\begin{array}{cc}R\cross{\angvel} & R\linvel\\
%%     0 & 0\end{array}\right)
%%     \left(\begin{array}{c} ^e\tool + ^{e}\hat{\Delta}_i\\ 1\end{array}\right)\\
%%       &=& R\cross{\angvel}(^e\tool + ^{e}\hat{\Delta}_i) + R\linvel\\
%%       \left(\begin{array}{cc}R & -R\cross{^e\tool + ^{e}\hat{\Delta}_i}\end{array}\right)&&
%% \end{eqnarray*}
$$
J = \left(\begin{array}{cc} ^{c}R_{e}& -^{c}R_{e}\cross{^e\tool +\ ^{e}\hat{\Delta}_i}\end{array}\right)
$$
where $^{c}R_{e}$ is the rotation matrix of $\transf{c}{e}$.
To compute $\transf{c}{e}$, we use Newton Raphson algorithm: we linearize the
equation around the current value of $\transf{c}{e}$ and solve the linear
system. Let us denote
\begin{eqnarray*}
  \x &=& \transf{c}{e} \\
  f(\x) &=& \left(\begin{array}{c}
    \x(^e\tool + ^{e}\hat{\Delta}_1) - \hat{^{c}\hole}^1\\
    \vdots\\
    \x(^e\tool + ^{e}\hat{\Delta}_N) - \hat{^{c}\hole}^N\\
    \end{array}\right) \\
    \frac{\partial f}{\partial \x}(\x) &=&
    \left(\begin{array}{cc}
      ^{c}R_{e} & -^{c}R_{e}\cross{^e\tool +\ ^{e}\hat{\Delta}_1}\\
      \vdots & \vdots \\
      ^{c}R_{e} & -^{c}R_{e}\cross{^e\tool +\ ^{e}\hat{\Delta}_N}
    \end{array}\right)
\end{eqnarray*}

Starting from $\x_0$ initialized with the initial value of $\transf{c}{e}$ as
defined in the urdf file of the robot, we iterate as follows:
$$
\x_{i+1} = \x_i - \frac{\partial f}{\partial \x}(\x_i)^{+}f(\x_i),
$$
where $.^{+}$ denotes the Moore Penrose pseudo-inverse. The algorithm stops when
the difference between two successive values of $\x$ is below a given threshold.

\subsection{Position of the tooltip in the end effector frame}

Each measurement $\hat{^{c}\hole}^i$ provides the measured position of the hole, and therefore of the tooltip in the frame of the camera. To get the position of the tooltip in the camera frame, we compute the average of these measurements:
$$
\bar{^{c}\hole} = \frac{1}{N}\sum_{i=1}^N \hat{^{c}\hole}^i
$$
And in the end-effector frame:
$$
\bar{^{e}\hole} =\; \transf{e}{c}\bar{^{c}\hole}
$$
\end {document}
