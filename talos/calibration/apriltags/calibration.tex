\documentclass {article}

\newcommand\conf{\mathbf{q}}
\newcommand\confoffset{\mathbf{q}_{off}}
\newcommand\transf{T}

\title {Calibration of Pyrene arms and camera}
\author {Florent Lamiraux}
\date {}
\begin {document}
\maketitle
\section {Introduction}

The objective of this note is to process data collected with Pyrene in order
to calibrate the sub-kinematic chain of the arms and the head, including the
position of Apriltags supports on the wrists and the position of the camera on the head.

\section {Position of the problem}

The data we try to evaluate are the following
\begin{itemize}
\item $\confoffset$ a vector of differences between the angular values of each joint ($\conf$) and the measured value ($\hat{\conf}$): $\conf = \hat{\conf} + \confoffset$,
\item $^{h}\transf_{c}$ the position of the camera in the head (joint \texttt{talos/head\_2\_joint})
\item $^{lw}\transf_{ls}$ the position of the tag support in the left wrist,
\item $^{rw}\transf_{rs}$ the position of the tag support in the right wrist.
\end{itemize}
Note that only components of $\confoffset$ corresponding to joints between the wrists and the head can be evaluated.

\subsection {Input data}

The input data consists of a list of $n$ measures of the form:
\begin{itemize}
  \item $m_i = (\hat{\conf}_i, ^{c}\hat\transf_{s\ i})$, $1 \leq i \leq n$,
\end{itemize}
where $s$ stands for left or right support depending on the wrist that has been detected.

\subsection {Constraints}

For each measure, we define a non-linear equation as follows:
$$
^{c}\hat{\transf}_{s\ i} = ^{c}{\transf}_{s} (\hat{\conf}_{i} + \confoffset)
$$
where the right hand side is the position of the support in the camera frame as
computed by the forward kinematics using the model of the robot.
$$
^{c}{\transf}_{s} (\hat{\conf}_{i} + \confoffset) = ^{c}{\transf}_{h}\ {\transf}_{h}^{-1}(\hat{\conf}_{i} + \confoffset)\ {\transf}_{w} (\hat{\conf}_{i} + \confoffset)\ ^{w}{\transf}_{s}
$$
\end {document}
